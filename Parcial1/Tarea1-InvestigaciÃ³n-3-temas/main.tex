\documentclass[a4paper, 12pt]{article}
\usepackage[utf8]{inputenc}
\usepackage[spanish]{babel}
\usepackage{url, hyperref}

\setlength{\parindent}{0pt}

\title{
    \vspace{-3cm}Tarea 1: Investigación de 3 temas de la robótica
    \author{
        Angel de Jesús Maldonado Juárez\\
        Universidad Autónoma de San Luis Potosí\\
        Facultad de Ingeniería - Ing. En Sistemas Inteligentes\\
        \textbf{Materia:} Programación de Robots\\
        \textbf{Prof:} Dr. César Augusto Puente Montejano\\
        \textbf{Autor:} Angel de Jesús Maldonado Juárez
    }
    \date{\textbf{Fecha de entrega:} martes 31 de enero de 2023}
}

\begin{document}
\maketitle

\hrule

\section*{Automóviles autónomos}
Los automóviles autónomos pueden definirse como un tipo de
vehículo que puede percibir el entorno por el cual transita,
y tomar decisiones o realizar maniobras con base en la
información que recupere de ese entorno, y siguiendo las leyes
o reglas de tránsito.

La Sociedad de Ingenieros en Automotores (\emph{SAE}) definen
6 niveles de automatización en el manejo de los coches, desde
$0$ (completamente manual) hasta $5$ (completamente autónomo):

\begin{enumerate}
    \item \textbf{Nivel 0 (completamente manual)}: El humano
          es el encargado de realizar todas las acciones de manejo.
    \item \textbf{Nivel 1 (asistencia al conductor)}: El
          vehículo brinda asistencia al conductor como control de
          velocidad, indicadores, etc.
    \item \textbf{Nivel 2 (automatización parcial)}: El
          vehículo realiza las tareas de giro y aceleración, y el
          humano sigue monitorizando las acciones y puede retomar el
          control en cualquier momento.
    \item \textbf{Nivel 3 (automatización condicionada)}: El
          vehículo tiene la capacidad de percibir el entorno, sin
          embargo, el humano aún debe realizar la monitorización de
          las acciones del vehículo.
    \item \textbf{Nivel 4 (alta automatización)}: El vehículo
          realiza todas las tareas de conducción bajo ciertas
          circunstancias y la geolocalización es requerida, la
          asistencia del humano ya es opcional.
    \item \textbf{Nivel 5 (automatización completa)}: El
          vehículo realiza todas las tareas de conducción en todas
          las circunstancias y escenarios, la asistencia del humano
          ya no es requerida.
\end{enumerate}
El modelo autónomo de \emph{Mercedes-Benz Clase S}, es considerado
el primer vehículo autónomo, este fue creado por el alemán Ernst
Dickmanns, en la década de los 90s. Este vehículo alcanzó un nivel
de autonomía del $95\%$.

Actualmente, la mayoría de coches implementan muchas herramientas
para asistencia de manejo a los conductores, pero pocas empresas
logran un nivel de autonomía completa (nivel 5) o alta automatización
(nivel 4), algunas de estas empresas son \emph{Google}, \emph{Tesla}
\emph{BMW}, \emph{Toyota}, \emph{Uber}, \emph{Apple}, \emph{Audi},
\emph{Volvo}, y \emph{Volkswagen}.

Los modelos de vehículos con mayor autonomía del mercado actualmente son:
\emph{BMWiX}, \emph{Tesla Model S}, y \emph{Mercedes EQS}.

\section*{Mars Perseverance Rover}
El \emph{Mars Perseverance rover} es una misión que forma parte del
\emph{Programa de Exploración de Marte de la NASA}, la importancia de esta
misión es lograr una exploración del planeta Marte para evaluar las
condiciones físicas para la posibilidad de habitar el planeta, o si
en algún tiempo en el pasado existió vida en el planeta, aunque haya
sido en forma de microorganismos.

El \emph{Perseverance Rover} puede ser considerado un robot, y algunas
de sus tantas características técnicas y físicas son que cuenta con
tecnología de punta para poder realizar las tareas de exploración más
importantes, como que cuenta con una cámara y un analizador a base de
láser para poder extraer las propiedades físicas y químicas de las muestras
de rocas y otros elementos recolectados por el robot. El chasis y los
componentes internos del robot fueron fabricados de manera que tuvieran
una tolerancia a altos niveles de radiación, también cuenta con una gran
cantidad de sensores y cámaras que son de utilidad tanto para los humanos
que reciben información de estos sensores y cámaras como para el sistema
autónomo del robot. La forma en que los científicos reciben información del
\emph{Perseverance Rover} es a través de radiofrecuencias mediante las
distintas antenas que tiene el robot, donde estas primero envían señales a
los satélites orbitales (\emph{Mars Orbiters}) y estos después retransmiten
las señales hacia la Tierra, tardando entre 5 y 20 minutos en llegar
dependiendo en la ubicación de los planetas. Las características principales
de la computadora interna (cerebro) del \emph{Perseverance Rover}, son que
tiene un procesador con arquitectura \emph{BAE RAD 750}, el cual puede
alcanzar velocidades de procesamiento de 200 megahertz. Y su memoria tiene
una capacidad 2 gigabytes de memoria flash, 256 megabytes de \emph{RAM},
y 256 kilobytes de memoria \emph{ROM}.
Muchas más características y especificaciones del \emph{Perseverance Rover}
pueden visualizarse en la página oficial de NASA \cite{mars.nasa.gov_2020},
donde se muestra un modelo en 3D interactivo del robot.

\section*{DARPA Robotics Challenge}
\emph{DARPA} (\emph{Defense Advanced Research Projects Agency}) es una agencia
de investigación y desarrollo en tecnología especializada para el uso militar, y
\emph{DARPA Robotics Challenge} es una competencia organizada por esta agencia
para que los participantes desarrollaran robots capaces de realizar tareas
que pudieran resultar peligrosas para los seres humanos bajo condiciones extremas.

Una de las competencias más representativas fue en 2015, en esa ocasión los retos
que el robot tenía que cumplir eran conducir un vehículo \emph{ATV}, subir una
escalera de 8 pies de altura, abrir una puerta de palanca con mango, una carrera
de obstáculos, cortar un agujero en una pared, hacer girar 3 válvulas de aire,
desenrollar una manguera y conectarla en un grifo, y limpiar los restos de una
puerta.

Los ganadores del premio, el equipo \emph{KAIST} (Corea del Sur) lograron superar
las pruebas en menos de 45 minutos, la característica principal del robot de
\emph{KAIST} (\emph{DRC-HUBO}), era que podía encogerse de tamaño para realizar
algunas tareas específicas, optimizando el espacio y dificultad de agacharse para
los robots con estructura humanoide. El segundo mejor robot, \emph{Atlas} fue
responsabilidad del equipo de \emph{Google}.

La importancia de los desempeños de los robots en esta competencia radica en sus
posibles aplicaciones, y \emph{DARPA} con la premiación y elección de los
participantes, abren un mundo de posibilidades en los cuales se pueden aplicar las
soluciones que implementaron los concursantes en sus robots para superar los
distintos retos, pudiendo salvar vidas humanas realizando tareas en las condiciones
extremas que suponen los escenarios de la competencia y en el mundo laboral de
algunos trabajadores con puestos de alto riesgo, como plantas nucleares, zonas de
mantenimiento en plantas industriales, entre otras.\\

\hrule

\bibliographystyle{plain}
\bibliography{refs}
\nocite{*}
\end{document}