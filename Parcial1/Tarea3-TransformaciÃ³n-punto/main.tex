\documentclass[a4paper, 12pt]{article}
\usepackage[utf8]{inputenc}
\usepackage[spanish]{babel}
\usepackage{amsmath}
\usepackage{listings, xcolor}

\definecolor{codegreen}{rgb}{0,0.6,0}
\definecolor{codegray}{rgb}{0.5,0.5,0.5}
\definecolor{codepurple}{rgb}{0.58,0,0.82}
\definecolor{backcolour}{rgb}{0.95,0.95,0.92}
\lstdefinestyle{mystyle}{
    backgroundcolor=\color{backcolour},   
    commentstyle=\color{codegreen},
    keywordstyle=\color{magenta},
    numberstyle=\tiny\color{codegray},
    stringstyle=\color{codepurple},
    basicstyle=\ttfamily\footnotesize,
    breakatwhitespace=false,         
    breaklines=true,                 
    captionpos=b,                    
    keepspaces=true,                 
    numbers=left,                    
    numbersep=5pt,                  
    showspaces=false,                
    showstringspaces=false,
    showtabs=false,                  
    tabsize=2
}

\lstset{style=mystyle}
\lstset{style=mystyle, language=Python}

\title{
    \vspace{-3cm}Tarea 3: Ejercicio de tranformación de un punto en
    el espacio
    \author{
        Universidad Autónoma de San Luis Potosí\\
        Facultad de Ingeniería - Ing. En Sistemas Inteligentes\\
        \textbf{Materia:} Programación de Robots\\
        \textbf{Prof:} Dr. César Augusto Puente Montejano\\
        \textbf{Autor:} Angel de Jesús Maldonado Juárez
    }
    \date{\textbf{Fecha de entrega:} martes 7 de febrero de 2023}
}

\begin{document}
\maketitle
\hrule\vspace*{1cm}

Dado el punto \(p=(7,3,1)\), aplicar las siguientes transformaciones:
\begin{enumerate}
    \item Rotación de \(90\) grados en el eje \(Z\).
    \item Traslación de \([4,-3,7]\).
    \item Rotación de \(90\) grados en el eje \(Y\).
\end{enumerate}

\textbf{Matriz de movimiento}
\begin{equation*}
    \begin{bmatrix}
        cos(90)  & 0 & sen(90) & 0 \\
        0        & 1 & 0       & 0 \\
        -sen(90) & 0 & cos(90) & 0 \\
        0        & 0 & 0       & 1
    \end{bmatrix}
    *
    \begin{bmatrix}
        1 & 0 & 0 & 4  \\
        0 & 1 & 0 & -3 \\
        0 & 0 & 1 & 7  \\
        0 & 0 & 0 & 1
    \end{bmatrix}
    *
    \begin{bmatrix}
        cos(90) & -sen(90) & 0 & 0 \\
        sen(90) & cos(90)  & 0 & 0 \\
        0       & 0        & 1 & 0 \\
        0       & 0        & 0 & 1
    \end{bmatrix}
    =
\end{equation*}

\begin{equation*}
    \begin{bmatrix}
        0  & 0 & 1 & 7  \\
        0  & 1 & 0 & -3 \\
        -1 & 0 & 0 & -1 \\
        0  & 0 & 0 & 1
    \end{bmatrix}
\end{equation*}

\textbf{Matriz de movimiento aplicada al punto}
\begin{equation*}
    \begin{bmatrix}
        0  & 0 & 1 & 7  \\
        0  & 1 & 0 & -3 \\
        -1 & 0 & 0 & -1 \\
        0  & 0 & 0 & 1
    \end{bmatrix}
    *
    \begin{bmatrix}
        7 \\
        3 \\
        1 \\
        1
    \end{bmatrix}
    =
    \begin{bmatrix}
        8   \\
        0   \\
        -11 \\
        1
    \end{bmatrix}
\end{equation*}

\textbf{Código en Python para obtener la solución}
\lstinputlisting{src/main.py}

\end{document}