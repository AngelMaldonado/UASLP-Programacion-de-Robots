\documentclass[a4paper, 12pt]{article}
\usepackage[utf8]{inputenc}
\usepackage[spanish]{babel}
\usepackage{amsmath}

\title{
    \vspace{-3cm}Tarea 4: Matriz de transformación para método D-H
    \author{
        Universidad Autónoma de San Luis Potosí\\
        Facultad de Ingeniería - Ing. En Sistemas Inteligentes\\
        \textbf{Materia:} Programación de Robots\\
        \textbf{Prof:} Dr. César Augusto Puente Montejano\\
        \textbf{Autor:} Angel de Jesús Maldonado Juárez
    }
    \date{\textbf{Fecha de entrega:} martes 14 de febrero de 2023}
}

\begin{document}
\maketitle
\hrule\vspace*{1cm}
\section*{Cadenas cinemáticas}
Un robot está compuesto por un conjunto de enlaces, conectados
por varias uniones. Estas uniones pueden ser simples o complejas.
La diferencia radica en que las simples solamente tienen un
grado de libertad (el ángulo de rotación o el desplazamiento lineal),
mientras que otros mecanismos pueden llegar a tener hasta tres
grados de libertad.

Un robot manipulador con \(n\) uniones tendrá \(n+1\) enlaces.
Se enumeran las uniones de \(1\) hasta \(n\), y se enumeran los
enlaces de \(0\) hasta \(n\). Utilizando esta convención, la unión
\(i\) conecta al enlace \(i-1\) para enlazar \(i\). \textbf{Cuando
    la unión \(i\) se acciona, el enlace \(i\) se mueve}. Sin embargo,
la unión \(0\) está fija, y no se mueve cuando las demás uniones se
accionan.

\section*{Método Denavit Hartenbert}
Es posible pasar de eslabón por eslabón mediante 4 transformaciones
básicas que dependen de las características geométricas de cada
eslabón. Estas transformaciones son:

\begin{equation}
    ^{i-1}A_i=R_z(\theta_i)H(0,0,d_i)H(a_i,0,0)R_x(\alpha_i)
\end{equation}

Calculando la matriz de transformación utilizando la convención de
D-H, quedaría:

\begin{equation*}
    A_i=
    \begin{bmatrix}
        c\theta_i & -s\theta_i & 0 & 0 \\
        s\theta_i & c\theta_i  & 0 & 0 \\
        0         & 0          & 1 & 0 \\
        0         & 0          & 0 & 1
    \end{bmatrix}
    \begin{bmatrix}
        1 & 0 & 0 & 0   \\
        0 & 1 & 0 & 0   \\
        0 & 0 & 1 & d_i \\
        0 & 0 & 0 & 1
    \end{bmatrix}
    \begin{bmatrix}
        1 & 0 & 0 & a_i \\
        0 & 1 & 0 & 0   \\
        0 & 0 & 1 & 0   \\
        0 & 0 & 0 & 1
    \end{bmatrix}
    \begin{bmatrix}
        1 & 0            & 0             & 0 \\
        0 & c_{\alpha_i} & -s_{\alpha_i} & 0 \\
        0 & s_{\alpha_i} & c_{\alpha_i}  & 0 \\
        0 & 0            & 0             & 1
    \end{bmatrix}
\end{equation*}
\begin{equation*}
    =
    \begin{bmatrix}
        c_{\theta_i} & -s_{\theta_i}c_{\alpha_i} & s_{\theta_i}s_{\alpha_i}  & a_ic_{\theta_i} \\
        s_{\theta_i} & c_{\theta_i}c_{\alpha_i}  & -c_{\theta_i}s_{\alpha_i} & a_is_{\theta_i} \\
        0            & s_{\alpha_i}              & c_{\alpha_i}              & d_i             \\
        0            & 0                         & 0                         & 1
    \end{bmatrix}
\end{equation*}

Las 4 cantidades \(\theta_i\), \(a_i\), \(d_i\), \(\alpha_i\) son parámetros
asociados con cada eslabón/enlace \(i\) y unión \(i\). Estos parámetros
también son llamados \textbf{longitud del eslabón}, \textbf{giro del enlace},
\textbf{umbral del enlace}, y \textbf{ángulo del enlace}, respectivamente.


\end{document}